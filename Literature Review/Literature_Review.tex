% use "amsart" instead of "article" for AMSLaTeX format
\documentclass[a4paper,11.5pt]{report}   
\renewcommand{\baselinestretch}{1.5} 

\usepackage[strings]{underscore}
\usepackage{datetime}

\newdateformat{monthyeardate}{%
  \monthname[\THEMONTH], \THEYEAR}

\usepackage{enumerate}

\usepackage{setspace}

%\usepackage[margin=1.2in]{geometry}

\usepackage[a4paper,bindingoffset=0.4in,%
            left=1in,right=1in,top=1in,bottom=1in,%
            footskip=.25in]{geometry}
            


% ... or a4paper or a5paper or
\geometry{a4paper}      
             		
% Activate to begin paragraphs with an empty line rather than an indent
\usepackage[parfill]{parskip}   

 % Use pdf, png, jpg, or eps§ with pdflatex; use eps in DVI mode		
\usepackage{graphicx, wrapfig}				

\setcounter{secnumdepth}{4}

\usepackage{nomencl}
\makenomenclature

\usepackage{lipsum}

\usepackage{gensymb}

%\usepackage[monochrome]{color}
%\usepackage[utf8]{inputenc}


\usepackage{float}

\usepackage{amsmath}
	\numberwithin{figure}{section}
	\numberwithin{table}{section}
	\numberwithin{equation}{section}

\usepackage{amsmath}
	\numberwithin{equation}{section}
	\newcommand*{\Scale}[2][4]{\scalebox{#1}{$#2$}}%
	
\usepackage{multirow}

\usepackage{appendix}

\usepackage{array}

%\newcolumntype{P}[1]{>{\centering\arraybackslash}p{#1}}

\usepackage{blindtext}

\usepackage[export]{adjustbox}[2011/08/13]
	
\usepackage{gensymb}
	
\usepackage{colortbl}

\usepackage{longtable}

\usepackage{tabularx}

\usepackage{enumerate}

\newcommand{\gray}{\rowcolor[gray]{.90}}

\makeatletter
\newcommand*{\rom}[1]{\expandafter\@slowromancap\romannumeral #1@}
\makeatother

\usepackage{rotating}

%\bibliographystyle{unsrtnat}
\usepackage{natbib}
\newcommand*{\urlprefix}{Available from: }
\newcommand*{\urldateprefix}{Accessed }
\bibliographystyle{bathx}
%\usepackage[sort&compress,numbers]{natbib}


\usepackage[final]{pdfpages}

\usepackage{caption}

\usepackage[% line break after label
   singlelinecheck=off, font=bf]{caption}
%\newcommand{\rom}[1]{%
%  \textup{\expandafter{\romannumeral#1}}%
%}

\usepackage{tocloft}
\renewcommand{\cftpartleader}{\cftdotfill{\cftdotsep}} % for parts
\renewcommand{\cftchapleader}{\cftdotfill{\cftdotsep}} % for chapters
 \renewcommand\cftchapafterpnum{\vskip10pt}

 
%\usepackage{titletoc}% http://ctan.org/pkg/titletoc
%\titlecontents*{chapter}% <section-type>
%  [0pt]% <left>
%  {}% <above-code>
%  {\bfseries\chaptername\ \thecontentslabel.\quad}% <numbered-entry-format>
%  {}% <numberless-entry-format>
%  {\bfseries\hfill\contentspage}% <filler-page-format>


\usepackage{subfig}					%use figure packagesx

\makeatletter

\newcommand\frontmatter{%
    \cleardoublepage
  %\@mainmatterfalse
  \pagenumbering{Roman}}

\newcommand\mainmatter{%
    \cleardoublepage
 % \@mainmattertrue
  \pagenumbering{arabic}}

\newcommand\backmatter{%
  \if@openright
    \cleardoublepage
  \else
    \clearpage
  \fi
 % \@mainmatterfalse
   }

\makeatother

\usepackage{fancyhdr}
\fancyhf{}
  \fancyhf[lef,rof]{\thepage}%
\pagestyle{fancy}
\fancypagestyle{plain}{%
  \fancyhf{}%
  \renewcommand{\headrulewidth}{0pt}%
  \fancyhf[rof]{\thepage}%
}

\usepackage{memhfixc}

\renewcommand{\contentsname}{Table of Contents}

\usepackage{pdflscape}

\usepackage{lscape}

\usepackage{textcomp}

%\usepackage[libertine,cmintegrals,cmbraces,vvarbb]{newtxmath}

%\usepackage[sorting=none]{biblatex}

\usepackage{notoccite}

\usepackage{titlesec}
\usepackage{appendix, apptools}
\AtAppendix{%
\titleformat{\chapter}[display]{\vspace*{-30pt}\bfseries\huge}{\chaptername~\thechapter}{1em}{}
\titlespacing*{\chapter}{0pt}{0pt}{0pt}}%

\makeatletter
\newcommand{\mypm}{\mathbin{\mathpalette\@mypm\relax}}
\newcommand{\@mypm}[2]{\ooalign{%
  \raisebox{.1\height}{$#1+$}\cr
  \smash{\raisebox{-.6\height}{$#1-$}}\cr}}
\makeatother

\usepackage{afterpage}
\newcommand\blankpage{%
    \null
    \thispagestyle{empty}%
    \addtocounter{page}{-1}%
    \newpage}
    
\usepackage{url}

\usepackage[official]{eurosym}


%\title{
%Development an Educational Game to Teach SQL Programming
%}
%\author{
%Amarnath Kakkar \\
%\normalsize University of Bath
%}
%\date{}							% Activate to display a given date or no date


\begin{document}

\frontmatter


%titlepage
\clearpage\thispagestyle{empty}
\begin{center}
\begin{minipage}{0.9\linewidth}
    \centering
%University logo
    \vspace{0.8cm}
    \includegraphics[width=0.4\linewidth]{logobath.jpg}\par
    
    \vspace{2.5cm}
%Thesis title
    \vspace{0.2cm}
    {\uppercase{\Large \textbf{Development of an Educational Game to Teach Iteration and Conditional} \par}}
    \vspace{3cm}
%Author's name
    {\Large \textbf{Amarnath Kakkar}\par}
        \vspace{3cm}
% $^{\circ}$
    {\large A final year project submitted in partial fulfilment for the degree of Bachelor's in Computer Science and Mathematics with Honours\par}
    {\large University of Bath\par}
    \vspace{4.5cm}
%supervisor and date
    {\large \textbf{\monthyeardate\today}\par}
    \vspace{1cm}
\end{minipage}
\end{center}

\afterpage{\blankpage}




\clearpage\thispagestyle{empty}
\begin{center}
\begin{minipage}{0.9\linewidth}
%University logo

\vspace{10cm}
{This dissertation may be made available for consultation within the University Library and may be photocopied or lent to other libraries for the purposes of consultation.\par}
\vspace{1cm}
Signed: 
   	
	
\end{minipage}
\end{center}

\newpage



    
\clearpage\thispagestyle{empty}
\begin{center}
\begin{minipage}{1\linewidth}
%Thesis title
%    \vspace{0.2cm}
%    {\uppercase{\large {Submitted in partial fulfilment for the degree of masters in aerospace engineering} \par}}
    \vspace{2cm}
%%Author's name
    {\LARGE Development of an Educational Game to Teach Iteration and Conditional \par}
    \vspace{1cm}	
    {\large Submitted by: Amarnath Kakkar\par}
	
    \vspace{1.5cm}
    {\Large \textbf{COPYRIGHT}\par}
    \vspace{0.5cm}
    {Attention is drawn to the fact that copyright of this dissertation rests with its author. The Intellectual Property Rights of the products produced as part of the project belong to the author unless otherwise specified below, in accordance with the University of Bath?s policy on intellectual property
(see http://www.bath.ac.uk/ordinances/22.pdf).
This copy of the dissertation has been supplied on condition that anyone who consults it is understood to recognise that its copyright rests with its author and that no quotation from the dissertation and no information derived from it may be published without the prior written consent of the author.\par}

     \vspace{0.5cm}
     {\Large \textbf{Declaration}\par}
      \vspace{0.5cm}
      {This dissertation is submitted to the University of Bath in accordance with the requirements of the degree of Bachelor of Science in the Department of Computer Science. No portion of the work in this dissertation has been submitted in sup- port of an application for any other degree or qualification of this or any other university or institution of learning. Except where specifically acknowledged, it is the work of the author.\par}
    
     \vspace{2.5cm}
%% $^{\circ}$
    {\large Department of Computer Science\par}
    {\large University of Bath\par}
    \vspace{0.5cm}
%supervisor and date
    {\large Supervisor: Dr. Alan Hayes}\par
    {\large \textbf{\monthyeardate\today}\par}
    \vspace{1cm}
\end{minipage}
\end{center}

\afterpage{\blankpage}

\hfill

\section*{\Huge{Abstract}}

To be written.



\newpage

\addtocontents{toc}{\protect\setstretch{1}}
\tableofcontents

\newpage
{%
\let\oldnumberline\numberline%
\renewcommand{\numberline}{\figurename~\oldnumberline}%
\listoffigures%
}

\newpage
\listoftables  

\newpage
\hfill

\section*{\Huge{Outline of Project}}

To be written.

\newpage
\hfill
\section*{\Huge{Acknowldgements}}

To be written.
 
\afterpage{\blankpage}

\mainmatter

\chapter{Introduction}

%use Breuer2010 to define game
%Platform: Personal Computer
%Subject Matter: Programming fundamentals - Iterative and Conditional Control Structures
%Learning Goals: Programming skills
%Learning Principles: Trial and Error
%Target Audience: First year university students
%Interaction Mode: Single Player
%Application Area: Academic Learning
%Controls/Interfaces: Mouse & Keyboard
%Common Gaming Tags: 2D Maze Adventure


\afterpage{\blankpage}

\chapter{Literature Review}
%\section{Aims of This Section}

%This Literature Review aims to bring together various studies, and provide an insight into the use of games for educational purposes. It also aims to critically evaluate the material for any consistencies or inconsistencies, and to hopefully provide another perspective on the field.

%\subsection{Defining Serious Games}
%In the area of gamification, various definitions of serious games are found. Most commonly, and perhaps quite broadly, the term 'serious game', is defined as, digital games used for purposes other than mere entertainment \cite{Susi2416}.  While a variety of definitions have been suggested, this dissertation will use the definition suggested by \cite{Garris2002} who saw it as "instructional games that are designed for training or to promote learning" together with "games that do not have entertainment, enjoyment or fun as their primary purpose"  \cite{MichaelChen2006}. However, it is not to say that serious games should not be entertaining, enjoyable or fun.

%\subsection{Outline of Review}
%This review will; identify the benefits of serious games, and make clear any negatives, go through examples of serious games and their approaches, including any specific game design elements implemented, and outline any techniques that seem to be of positive influence or negative influence, look into existing resources for teaching SQL and outline the trade-offs with each approach, finally bring together these ideas summarise various points.

%\section{Benefits of Serious Games}
%Today's students are brought up in the digital era. \cite{Prensky2001} refers to them as 'Digital Natives', that have experienced a new form of video game play, which brings opportunities with great potential for their learning. Thus potentially making games great learning environments within education. However, \cite{Girard2013} summarises that there is need for more empirical research to determine the effectiveness of serious games. On the other hand, \cite{Pieter2013} reports that in general, serious games are more effective than traditional teaching methods, and that more studies are required to determine the effects of various features.

%In this section, we will outline some of the empirical researches that exist on the internet.

%\subsection{}

%\section{Examples of Serious Games}
%\subsection{Game Design Elements}
%\section{Existing Methods for Teaching SQL}
%\section{Summary}


\section{Video Games}

%The academic study of games, ludology, is a relatively new and emerging field. As the video game revolution took off in the late 20th century, so did academic interest in games. 

%Video games for entertainment have dominated the market, 

%The academic study of games, is a relatively new and emerging field. Games are naturally creative things, however academics see the potential in games for a variety of different reasons, and so game design frameworks have been developed to help design and create games.

The video games industry has grown very rapidly in recent years, and is expected to continue to grow. The current value of the market is predicted as \$150 billion USD and is expected to reach \$180 billion by 2022 \citep{vgamesResearch}. In 2006, video games were considered as one of the most popular forms of entertainment in the United States \citep{sherry2006, ritterfeld2006}. Now video games can be considered a popular form of entertainment globally.

%At one time, particu- larly in the 1970s, the term video games meant ?games played in a video arcade.?

\citeauthor{Botturi2009} claimed that video games in the 1970s meant "games that were playeable in amusement arcades" \citep{Botturi2009}. Since then, a video game can be defined as "a mental contest, played with a computer according to certain rules for amusement, recreation, or winning a stake" \citep{Zyda2005}.

%Military applications have dominated the market, but civil applications are expected to grow [8][9]. UAV have already proved to be successful in field operations however, further research can enable UAV?s to be used in intelligence gathering, such as stealth and combat operations [34].

\subsection{History}

%The interest in UAV?s has been observed since 1916, when the first modern unmanned aircraft was invented, Hewitt?s UAV. This was a result of Sperry?s work, on the flight stabilisation using gyroscope devices, which provided flight stabilisation [11]. This attracted the interest of the US Navy however, due to technical difficulties the research and work in automatic planes was lost. In 1933 the Royal Navy Queen bee?s target drone was operated for the first time and the potential of UAV?s was understood, but it still required perfection of remote operations. Reginald Denny then developed the successful target drone RP-2, during WWII using radio control [12].

The earliest documented predecessor to video games was observed in 1948, when the "Cathode-Ray Tube Amusement Device" was patented. The amusement device, required players to overlay pictures of targets such as airplanes in front of the screen \citep{thefirstvideogame}.

10 years later, physicist William A. Higinbotham was credited for creating the first video game; attempting to display his research at an exhibition, he anticipated that his display would not generate any interest, so he conceptualised and created 'Tennis for Two' \citep{TennisForTwo}. Tennis for Two was created using an analog computer with an oscilloscope for a screen. It was the first game to display motion and allow multiple players to play together \citep{thefirstvideogame}. 

%Computer games are defined as games that are played on Personal Computers, and video games as games played using a television and a games console \cite{Cummings07}.

%In 1958, Tennis for Two was created using an analog computer and oscilloscope for a screen. Four years later, Spacewar was developed using a digital minicomputer and a cathode-ray tube as the display, making it one of the first computer games. In 1972, Computer space and Pong were among the first video games. They were played on televisions placed in upright cabinets, and this paved the look and feel for future arcade games \cite{TennisForTwo, Lowood2009}.

The rise of modern generation of video games is credited to the development of 'Spacewar!' in 1962, 'Computer Space' in 1971 and 'Pong' in 1972. Spacewar! was developed for academic purposes to test the limits of new hardware, but shortly after became very popular. Spacewar! was played by Nolan Bushnell, who used the idea of the game to create Computer Space, although, Computer Space did not gain much popular traction. This was partly due to its long winded instructions and complex game controls. Learning from these mistakes, the creators of Computer Space decided to create a simpler game and came up with the idea for Pong, which became very popular. Computer Space and Pong were designed solely for entertainment, and Pongs popularity was credited to the simplicity of its design \citep{Lowood2009}.


%The rise of modern generation of video games was a through the development of 'Spacewar!' in 1962, 'Computer Space' in 1971 and 'Pong' in 1972 \cite{Hector2003}. These video game were designed solely for entertainment \cite{Jean}. Spacewar! defined the main streams of computing research. Computer space did not gain traction, due partly to its long winded instructions and complex game controls, and so learning from these mistakes, the creators of Computer Space created the popular game 'Pong', and Pong's triumph was credited to the simplicity of its design \cite{Lowood2009}.

In the late 1980s, video games became a mainstream media industry \citep{Dmitri2003}.

\subsection{Impacts}


Initially, the majority of research on the effects of playing video games focused on the negative impacts, such as the potential aggression, addiction and depression from 'gaming'. But recently, researchers have argued that a more balanced perspective is needed \citep{Granic2014}. Studies have now also argued against the potential correlation between aggression and violent video games \citep{Ferguson2007}. Playing video games has also been linked to an increase in  perceptual, cognitive, behavioural, affective and motivational abilities \citep{Connolly2012}.


\subsection{Uses}

Video games are now used for a wide variety reasons. They are becoming ever more important in the global education and training market. Aside from entertainment, they are being used in: military, government, education, corporate and healthcare \citep{Johann2015}.

%Many attempts to pinpoint the creation of the very first video game have been made. However, the creation of the first modern video game is commonly dated to 1958. first video game ever invented was developed by a physicist William Higinbotham, in 1958. The game was called 'Tennis for Two'. The design and concept was conjured up from an instructional book the inventor was reading at the time \cite{TennisForTwo}.

%The first recorded video game invented, was developed by a physicist called William Higinbotham, called 'Tennis for Two' in 1958. The concept of the game, came from an instructional book that he was reading at that time \cite{TennisForTwo}.

%In 1970, R. Barton discusses the evolution of 'The Imaginit Management Game' and the development of a game model. This model consisted of a set of rules, such that a management game could be repurposed to fit another management scenario \cite{Barton1970}.

%First a set of generality aspirations were defined, and then a set of strategies 'describes features of "generalized" management computer game and then reports a case history of adapting this game model, which was designed for ideal generality, to an application that challenged that very generality.

%\section{Benefits of Video Games}

%\subsection{Platforms}

\section{Educational Games}

The term 'Serious Game' can be used to describe an educational game; that is a game that has an educational purpose and is not intended to be played primarily for entertainment \citep[see][]{abt1970}. 

Serious games became an established academic field of study in 2007, by the founding of The Serious Games Institute \citep{Wilkinson2016}. The market value of the serious games industry in 2016 was predicted at \$1.5 billion USD, and is predicted to reach \$9 billion in 2023 \citep{alliedmarketresearch}.

\subsection{Definition}

There is currently no singleton definition for term 'Serious Game'. \citeauthor{Johann2015} argue that, groups and individuals define the term depending on their perspectives and interests, and that there are a wide variety of groups and individuals focusing on different issues \citep{Johann2015}. The first recorded definition of the term was set out in 1970 by \citeauthor{abt1970} \citep{Wilkinson2016}, who defined it as follows: "Games that have an explicit and carefully thought-out educational purpose, and are not intended to be played primarily for amusement. This does not mean that serious games are not, or should not be, entertaining." \citep{abt1970}. In \citeyear{Michael2005}, \citeauthor{Michael2005} re-interpret this definition to "Games that do not have entertainment, enjoyment or fun as their primary objective" \citep{Michael2005}. Thus suggesting that serious games are not limited to only educational purposes. The commonly agreed upon definition closely matches the definition by \citeauthor{Michael2005} \citep[see][]{Johann2015}.

In the same year \citeauthor{Michael2005} defined the term, \citeauthor{Zyda2005} provided his own. There is however a contradiction between these two definitions \citep{Johann2015}. Whilst \citeauthor{Michael2005} say that serious games should not have entertainment or fun as their primary objective, \citeauthor{Zyda2005} says that the entertainment component of the game should come first, and that the story of the game is more important than the pedagogy \citep{Zyda2005}. His definition is as follows: "a mental contest, played with a computer in accordance with specific rules, that uses entertainment to further government or corporate training, education, health, public policy, and strategic communication objectives" \citep{Zyda2005}. However this definition also suggests that serious games can only be digital \citep{Jean}. 

For the purpose of this dissertation, I will use the definition provided by \citeauthor{abt1970} and work entertainment around the primary purpose of the game - to teach.

%"Educational game" is still a newly emerging thing in our country, and there is no explicit definition nowadays. Narrowly speaking, educational game refers to the integration of education and game, and the education effect naturally generated from the process of playing games, in other words, it means "a type of computer game software which generates education effect through interest" \cite{song2008}



%There is a difficulty defining the term ?serious game?, as there appears to be a contradiction between its constituents terms; ?serious? and ?game? seem to be mutually exclusive \cite{Johann2015}. 


%The first constituent, ?serious?, is according to Ben Sawyer (in Michael and Chen, 2006) intended to reflect the purpose of the game, why it was created, and has no bearing on the content of the game itself. Regarding the second constituent, already Wittgenstein (1953) showed that there are difficulties in defining the concept of a game. There simply are no necessary and sufficient conditions \cite{Johann2015}.


%\subsubsection*{Instructional Games} %get rid of these when finished

%An instructional game is defined as "a type of software function designed to increase motivation by adding game-like rules and/or competition to a learning activity" \cite{Roblyer2013}.

%However, \citeauthor{AtsusiHirumi2010}, define an instructional game as "an interactive, digital game (e.g., adventure, strategy, role-play, action, and massive multiplayer online games) that is designed specifically to facilitate learning" \cite{AtsusiHirumi2010}.

%\citeauthor{Hays2005} defined instructional games , instruction must be designed to support specific instructional objectives, which are determined by job requirements. Second, instruction must include the opportunity for a learner to interact with the instructional content in a meaningful way. Third, the student's performance must be assessed to determine if he or she has learned what was intended. Finally, the results of the assessment must be presented to the student in a relevant and timely manner to either reinforce correct actions or to provide remediation for incorrect actions.

%\subsubsection*{Serious Games} %get rid of these when finished

%There are many definitions for the term 'Serious Game', but most agree on a core meaning that serious games are (digital) games used for purposes other than mere entertainment. \citeauthor{Zyda2005}, defined them as "a mental contest, played with a computer in accordance with specific rules, that uses entertainment to further government or corporate training, education, health, public policy, and strategic communication objectives" \cite{Zyda2005}.
%So for the purpose of this dissertation, we will focus on the subset of serious games that are concerned with educational purposes.

%Video games for entertainment purposes have dominated the market, however serious games are expected to grow. Serious games have proved to be successful, however further research is required to determine the effectiveness of such games \cite{Susi2416}.

%This dissertation aims to create a game for learning in educational contexts. Thus this game fits all the above definitions.

\subsection{History}

Educational games have arguably existed since the 7th century. Among the oldest is the board game 'Chaturgana', which is argued by historians to be the precursor to chess \citep{Wilkinson2016}. The aim of the game was to teach officers to become better planners for battles \citep{Wilkinson2016}. Another board game created more recently - in the 20th century was 'Landlord's Game'; a precursor to monopoly. It was designed to illustrate the dangers of capitalist approaches to land taxes and property renting \citep{Wilkinson2016}. So we can see that, games designed to educate, have existed for a long time.


%The interest in UAV?s has been observed since 1916, when the first modern unmanned aircraft was invented, Hewitt?s UAV. This was a result of Sperry?s work, on the flight stabilisation using gyroscope devices, which provided flight stabilisation [11]. This attracted the interest of the US Navy however, due to technical difficulties the research and work in automatic planes was lost. In 1933 the Royal Navy Queen bee?s target drone was operated for the first time and the potential of UAV?s was understood, but it still required perfection of remote operations. Reginald Denny then developed the successful target drone RP-2, during WWII using radio control [12].

%During the Cold war, the development in reconnaissance missions increased and the first recon- naissance UAV was developed, called the MQM-57 Falconer [13]. Not long after, the Ryan Model 147 was launched, which was the first unmanned aircraft which is known as an UAV today. In conclusion, the importance and usefulness of UAV was demonstrated over the years and is now being further researched, focusing on longer endurance UAV?s and MAV?s [12].

%Interest in flying wing designs for both UAV?s and larger scale civil applications, are now being revisited.

The interest in digital educational games, has been observed since 1967, when the first educational program was developed, 'Logo Programming' \citep{hayes2008} . Logo programming was an environment that allowed players to utilise the programming language LOGO in order to learn mathematics \citep{feurzeig1969}. Logo became popular among schools in the US \citep{lehrer1986}. It also became a key part in educational games and educational strategies research \citep{hayes2008}. 

Academic interest that games that could be used could be used for purposes other than mere entertainment, was first noted in \citeyear{abt1970} by \citeauthor{abt1970}, in his book \textit{Serious Games} \citep{Breuer2010}. The rise in digital games around this time created an opportunity for developing serious games \citep{Wilkinson2016}. 

However, serious games did not gain much traction until 2002, when a game developed by the US army, became hugely popular. 'America's Army' was developed as a training and recruitment game for the military. It is now considered as the forefront of modern serious games \citep{Zyda2005, Wilkinson2016}. In conclusion the potential of using games as educational tools has been demonstrated for a long time, and is now being further researched and developed, focusing on training and education in a number of different industries \citep{Wilkinson2016}.

%In 1971, 'The Oregon Trail' was released. The Oregon Trail was created by history teachers, to cover American history in 1848 \citep{Jean}. This game became hugely popular, and now has several sequels and spinoffs. The rise in arcade games and home consoles around this time, created an opportunity for developing serious games \citep{Wilkinson2016}.

 
%Learning in games appears indisputable considering recent studies, but when it comes to the question of what and how players learn through playing games, controversial answers can be found \citep{Mitgutsch}.

%Around this same time, the power and potential of computer games for education and training was beginning to be uncovered \citep{neil2005}. Computer games were hypothesised to provide multiple benefits, such as: 

%\begin{itemize}
%\item Complex and diverse approaches to learning processes and outcomes
%\item Interactivity
%\item Ability to address cognitive as well as affective learning issues
%\item Motivation for learning
%\end{itemize}

%America's Army
%\subsection{Benefits}


%Computer games are defined as games that are played on Personal Computers, and video games as games played using a television and a games console \citep{Cummings07}.

%In 1958, Tennis for Two was created using an analog computer and oscilloscope for a screen. Four years later, Spacewar was developed using a digital minicomputer and a cathode-ray tube as the display, making it one of the first computer games. In 1972, Computer space and Pong were among the first video games. They were played on televisions placed in upright cabinets, and this paved the look and feel for future arcade games \citep{TennisForTwo, Lowood2009}.

%Computer games separated from video games in the early 1990s. Since then, 3D home consoles like the Sony Playstation and the Sega Saturn have been introduced. Some innovations to consoles include; touchscreen and motion control \citep{Cummings07}. Recently, we have seen the development and the use of Virtual Reality consoles for gaming, entertainment and learning \citep{vrhaptics}.

\subsection{Benefits}

Serious games have become an interesting area for multidisciplinary academic research \citep{Breuer2010}. There are interests from fields such as psychology, computer science, pedagogy, sociology and cultural studies \citep{Breuer2010}. Many studies have looked into and discussed the benefits of serious games in educational contexts, and I will discuss some of these below.

\subsubsection{E-learning}

E-learning can be defined as an approach to teaching and learning, based on the use of electronic media and devices \citep{sangra2012}. Thus, digital educational games can be seen as a type of e-learning.

Educational games have inherent beneficial properties. For instance, they are able to provide information on demand and just in time, and in the context of actual use and people's purposes and goals, something that does not often happen in schools \citep{Gee2003}.

Other properties of e-learning include: ease of accessibility; can be used in absence of teachers or instructors; provide opportunities for relations between learners, helping eliminate the potential of hindering participation; low cost per person served; allows self-pacing - allowing student to study at their own pace; high level of interactivity; ability to use attractive graphics, and is an engaging and entertaining activity \citep{arkorful2015, Girard2013}.

In a study carried out on what university students thought about e-learning, students reported that they expected e-learning to be an integral part of the learning process within higher education \citep{Connolly2012}. Thus the use of an educational game in higher education may not be that alien to students.

%Games are able to provide information on demand and just in time, not out of the contexts of actual use or apart from people's purposes and goals, something that happens too often in schools. People are quite poor at understanding and remembering information they have received out of context or too long before they can make use of it \cite{Gee2003}. 

%Games allow players to be producers and not just consumers. Along with the designer, the player's actions co-create the game world \cite{Gee2003}.

\subsubsection{Learning Through Games}


Games can be a great learning environment \citep[see][]{prensky2003, Gee2003}. Despite the vast research on the negative impacts of gaming, playing video games have been argued to foster a host of different skills, such as, visual attention, spatial skills, problem solving skills and creativity \citep{Granic2014}. 

Conventionally when starting out on a new game, players first need to learn the rules and the controls of the game, and then use this newly acquired knowledge to complete objectives or levels. As players progress through the game the objectives require increasingly complex solutions, which in turn tests the player's knowledge and skill. \citeauthor{vygotsky1978} coined the term \textit{the zone of proximal development}, where learning occurs when people are presented with tasks which are just beyond their current level of ability but may require some help to complete \citep{vygotsky1978}.

Today's leaners have grown up immersed in digital technology \citep{Prensky2001}; \citeauthor{Prensky2001} calls this generation of people "Digitial Natives". They have spent long periods of time playing video games \citep{prensky2003}. Today, 2.5 billion people actively play video games worldwide \citep{Statista}, and of these, 57\% are aged between 10 and 35 \citep{Statistanewzoo}. Therefore it seems natural to assume that this generation will be more receptive to computer-based learning \citep{Girard2013}.

According to \citet{Prensky2001}, playing games is a fun and pleasurable activity, and research has shown that fun and enjoyment are an important part of the learning process as learners can be more motivated and willing to learn \citep{Bisson1996, Cordova1996}. Games which also provide an optimal balance of challenge and frustration, leave players in a motivating state to continue to play \citep{Gee2003}.

%This motivational ?sweet spot? balances optimal levels of challenge and frustration with sufficient experiences of success and accomplishment

Evidence of learning through educational games has been demonstrated \citep{Connolly2012, Pieter2013, Girard2013}. However, the effectiveness of educational games is undetermined \citep{Connolly2012, Girard2013}. There are also different viewpoints when it comes to determining their effectiveness. Two such meta-analyses; \citet{Pieter2013} and \citet{Girard2013}, looked at the effectiveness of learning through games, and arrived to different conclusions. These studies are compared in Table \ref{tab:effectivess learning comparison}.

\begin{table}[H]
\centering
\caption{Comparison of meta-analyses on the effectiveness of learning through games}
\label{tab:effectivess learning comparison}
\begin{tabular}{p{3.4cm} | p{5cm} | p{5cm}}
       & \textbf{\citet{Pieter2013}}                  & \textbf{\citet{Girard2013}}               \\ \hline
\textbf{Types of Games}  & Serious games                & Serious games \& video games               
\\ \hline
\textbf{Measuring}   & Learning, retention and motivation                   & Learning and engagement              
\\ \hline
\textbf{Learning Outcomes}	& Knowledge or skill acquisition 	& Knowledge or skill acquisition
\\ \hline
\textbf{Instructional Domain}	& Biology, maths, language or engineering	& Various (including: academic knowledge, cognitive skills, professional knowledge, cancer therapies)
\\ \hline
\textbf{Studies Published Between}		& 1990 - 2012		& 2007 - 2011
\\ \hline
\textbf{Experimental Design of Studies}		& Posttest or pretest-posttest		& Atleast pretest-posttest
\\ \hline
\textbf{Studies Reviewed} & 39                         & 9		                 
\\ \hline
\textbf{Age Range of Studied Population} & Wide range                         & 9 - 47		                 
\\ \hline
\textbf{Results support effective learning through games} & Yes                         & Undertermined
\end{tabular}
\end{table}


\citeauthor{Pieter2013} found that serious games were more effective in terms of learning and retention when compared to conventional teaching methods, but not more motivating. Whereas \citeauthor{Girard2013} found that only a few of the games resulted in improved learning, with the others having no difference when compared to traditional methods of teaching. A point to note is that in \citet{Pieter2013}, serious games were compared to lectures, reading, drill and practice, or hypertext learning environments, whilst in \citet{Girard2013}, they were compared to face-to-face lessons, pencil-and-paper studying or no studying at all. The former is a more modern way of teaching, whereas the latter is very limited.

\citeauthor{Pieter2013} evaluated a broad spectrum of studies, whereas \citeauthor{Girard2013} only evaluated randomised control trial studies. \citeauthor{Pieter2013} argued that if they only considered the studies with randomised samples with a pretest-posttest design, similar to \citeauthor{Girard2013} study, the positive effects in favour of serious games may disappear. 

In conclusion, there is a need for more empirical research to determine the effectiveness of serious games, and this is starting to be addressed \citep{Connolly2012}. Though, a number of studies agree that serious games, support learning, and used alongside other instructional methods, can make learning more effective and can improve the learning experience \citep{Pieter2013, Concannon2005, Granic2014}


%\citeauthor{Girard2013} evaluated 6 serious games, and of these 6, \citeauthor{Pieter2013} evaluated 3 of them.


%\citeauthor{Pieter2013} compared serious games to more modern instructional methods, such as lectures, reading, drill and practice, or hypertext learning environments, whilst \citeauthor{Girard2013} compared them to more traditional methods, such as face-to-face lessons or pencil-and-paper studying.

%\citeauthor{Pieter2013} found that serious games were more effective in terms of learning and retention, but they were not more motivating than conventional instruction methods. This study also measured the effectiveness of using serious games in conjunction with other instructional methods, and this was found to be the most effective means of learning \citep{Pieter2013}.

%Whilst \citeauthor{Girard2013} found that only a few of the games resulted in improved learning, with the others having no difference when compared to traditional methods of teaching. The study concluded that it was impossible to draw any conclusions on the effectiveness of serious games.



%\citep[see][]{Pieter2013, Girard2013}


%However, despite  in a meta-analysis on the effectiveness of serious despite the potential of using games in educational environments, studies do not  \citep{Mitgutsch, Connolly2012}.

%In a meta-analysis on the effectiveness of learning and engagement of games for learning, \citeauthor{Girard2013} covered six serious games and five video games. Finding that 2 out of the 6 supported having a positive impact on learning compared to other types of training or no training, 1 showed no difference and 3 had no beneficial effect on learning \citep{Girard2013}. This study however looked at a small number of serious games. They also could not draw a conclusion about the effectiveness of serious games, arguing that there is a wide variety of types and uses for serious games.

%Analyses show that games promote learning. Games can support the development of a number of different skills; analytical and spatial skills, strategic skills and insight, learning and recollection capabilities, psychomotor skills, visual selective attention.


%ELSPA, 2006; Gee, 2005, 2007; Klopfer et al., 2009; Robertson, 2009; Shaffer, 2007

%Kirkpatrick?s four levels for evaluating training

%Baker and Mayer?s CRESST model of learning

%\subsubsection{Engagement}

%Prensky, 2001

%Good games operate at the outer and growing edge of a player's competence, remaining challenging, but do-able, which is a very motivating state for human beings \citep{Gee2003}.

%In a meta-analysis conducted by \citeauthor{Connolly2012} on the positive impacts of games, they found that students reported enjoying game-based approaches to learning and that they find them motivating \citep{Connolly2012}.

%\subsubsection{Feedback}

%\citeauthor{Gee2008} argued that people learn best from their experiences when they get immediate feedback during those experiences so that they can recognise and assess their errors and see where their expectations have failed \citep{Gee2008}. As an example, Cameron and Dwyer (2005) (as cited in \cite{Connolly2012}) found that including feedback into educational game about the accuracy of users answers, lead to improved performance in terms of knowledge acquisition.


\subsection{Examples}

Educational games for teaching computer programming (6)

(Not my words) Robocode (2001) is one of the first environments developed as an open source educational game in order to support java programming

\subsubsection{Gaming Platforms}

Between 1997 and 2007, roughly 90\% of serious games were developed for personal computers, the other 10\% were made for a number of different home gaming consoles \citep{ratan2009}.








\section{Teaching Programming Fundamentals}

In 1978, ACM Computing Curricula used the terms "CS1" and "CS2" to designate the first two courses in the introductory sequence of a computer science undergraduate course. CS1 described introducing students to programming fundamentals and CS2 to teaching data abstraction/data structures. The general principles of CS1 and CS2 have continued, but through the past years, the concepts covered in these courses have changed. The most recent curriculum by ACM-IEEE was created in 2013 \citep{Hertz2010}. This curriculum notes that the "vast majority of introductory courses are programming-focused, in which students learn about concepts in computer science (e.g., abstraction, decomposition, etc.) through the explicit tasks of learning a given programming language and building software artifacts". The curriculum considers 'conditional and iterative control structures' as fundamental programming concepts \citep{acm}. Thus these concepts are likely be taught as a part of CS1 or introductory programming units. %CS course guidelines

%Computer Science departments in the UK and USA report declining enrollment and high attrition rates on their degree programs. Student attrition is exacerbated during and between the first and second years of these programs. These high attrition rates are associated with considerable failure rates in introductory programming courses. The changes to the computing industry associated with, inter alia, new technologies, and the problems associated with learning to program, have led to Higher Education Institutions facing the pressing need to rethink their CS curricula, with special attention given to redesigning the CS1 course.

Conventional introductory programming courses at University are structured courses based on lectures and practical laboratory work, and a curriculum focused largely on programming knowledge - particularly relating to the features of the programming language being taught and how to use them \citep{Robins2003}. \citeauthor{Robins2003} suggest that this approach is popular, due to the important role of such programming knowledge in programming and the sheer volume and detail of language related features that can be covered \citep{Robins2003}.

Another method for learning programming fundamentals which are becoming increasingly popular, are online resources. These resources include; tutorial websites such as Codeacademy and Khan Academy, which have accumulated millions of users; block-based programming environments such as Scratch and Alice, which provide creative visual environments, and educational games \citep{Lee2015}. 

In comparison, traditional methods such as face-to-face or pencil-and-paper teaching \citep{Girard2013}, centers on instructors who have control over class content and the learning process, whereas, online learning, offers a learner-centered, self-paced learning environment. Online resources are also time and location flexible and provide unlimited access to learners \citep{Zhang2004}. On the other hand, whilst there are many benefits to online learning, there are doubts over its effectiveness \citep{Zhang2004}. Some argue that online learning should not replace traditional forms of learning \citep{Zhang2004, Gunasekaran2002, agal2010}, instead it should be used to complement the learning process \citep{Zhang2004} and potentially improving the quality of the learners education \citep{Concannon2005}. Whilst other researchers highlight the great advantages of games over traditional methods \citep{Girard2013}.

%A meta-analysis conducted in 1992, looked at the empirical research on the instructional effectiveness of games to conventional classroom instruction.

In conclusion, there are advantages and disadvantages to both methods of teaching. However, there is a lot of support that games are effective learning tools \citep{Girard2013}.

\subsection{Issues}

The complexity of teaching introductory programming, which includes; iteration and conditional statements, is widely acknowledged among educators \citep{Koulouri2014}. Novice programmers have difficulty in tracing (a method of mentally simulating the execution of the code before compiling), reading and understanding pieces of code and fail to grasp basic programming principles and routines. The overhead of learning the syntax and semantics of a language at the same time, and difficulties in combining new and previous knowledge and developing their general problem-solving skills, all add to the complexity of learning how to program \citep{Koulouri2014}. Therefore careful consideration will be taken to reduce these problems, with efforts to; introducing programming concepts at a reasonable pace, and making the game fun and enjoyable to keep the user motivated. %Problems with teaching introductory programming to novices.

Researchers have proposed guidelines as to what makes good introductory programming units. For example, \citeauthor{Stevenson2006} analysed assignments from textbooks and historical usage to look for students problems, and proposed a criteria as to what would make good programming assignments: (1) be based on real world problems, (2) allow students to generate realistic solution, (3) allow students to focus on current topic(s) within context of a larger problem, (4) be challenging, (5) be interesting, (6) make use of one or more APIs, (7) have multiple levels of challenge and achievement and finally (8) allow for some creativity and innovation \citep{Stevenson2006}.






\section{Educational Game Design}

When researching the effects and effectiveness of digital games for learning, the importance of enjoyment for/in education needs to be taken into account. This means that when the effectiveness of a serious game is assessed, the question about its entertainment value should always be addressed \citep{Breuer2010}.

Despite the similarities between games and learning, it is not sufficient to just assume that all forms of games are equally suitable for learning and that simply presenting material in a game-like setting will increase the quantity and quality of learning \cite{Breuer2010}.

Given this background, the ideal educational game combines entertainment and learning in a way that the players/learners do not experience the learning part as something external to the game. This idea of ?stealth learning? should inform any approach to designing, using and evaluating (digital) games for prescribed educational aims \cite{Breuer2010}.

The design and production of video games involves aspects of cognitive psychology, computer science, environmental design, and storytelling, to name a few \citep{Koster2004}.

Designing educational games requires a focus that is different from general game design; otherwise, we may end up designing fun games with little or no learning value \citep{Barnes2007}.

\citeauthor{Driskell2002} describes a "tacit model that is inherent in most studies of instructional games". The model is as follows. Initially, we define a set of learning outcomes and objectives that we wish to achieve. We then design an instructional program which incorporates certain characteristics of games, that delivers the desired learning objectives. Subsequently, the program triggers a cycle that includes user judgments, user behaviours and system feedback. If the pairing of the instructional content with the appropriate game features is successful and effective, the cycle achieves recurring and self-motivated game play. Finally, this engagement in the game leads to the achievement of the learning outcomes \citep{Driskell2002}. This model is illustrated in Figure \ref{fig:Input_process_outcome_game_model}.

\begin{figure}[H]
 \centering
    \includegraphics[width=1\textwidth]{Inputprocessoutcomegamemodel}
       \captionsetup{justification=centering}
\caption{Input-Process-Outcome Instructional Game Model  {\citep{Driskell2002}}}
\label{fig:Input_process_outcome_game_model}
\end{figure}


Pleasure/fun and challenge have been found as some of the key reasons for people playing entertainment games \citep{Connolly2012}.


\subsection{History}

In 1980, Thomas W. Malone had begun researching the natural motivational properties of video games. With this work Malone created a set of heuristics based on motivational principles of games to create fun instructional game designs \citep{Malone1987}.  In 1996, Lloyd Rieber Z

%Rieber, and others [100] argued that his fantasy element leads to engagement with the learning content. Here then, in the work of Rieber we have a logical extension of the approach of Malone ? that is the segmenting of ?games? into a set of heuristics that can be used to form Serious Games design.

%This work to unpack, stratify demarcate, or otherwise categorize games into a set of heuristics or design principles to be applied to educational games was a significant focus for researchers at the time [101]. Moreover, when reviewing the modern trends in Serious Games research, it is apparent that this trend has continued ? though the models are now more formalised [38, 102?104]

\subsection{Frameworks}

\subsubsection{Motivational}

\subsubsection{Learning}

To be written.

EFM: A Model for Educational Game Design

Game object model version II: a theoretical framework for educational game development (318)

Game, motivation, and effective learning: An integrated model for educational game design (230)

Serious Games: A New Paradigm for Education?

\subsection{Game Genre}

\citeauthor{Connolly2012} finds that simulation games are one of the most common game genres in serious games, possibly because their use in education is already established \citep{Connolly2012}. Simulation games have the ability to represent real-life situations \citep{Braghirolli2016}. 

https://www.researchgate.net/publication/272166351_A_taxonomy_of_educational_games

To encourage the use of games in learning beyond simulations and puzzles, it is essential to develop a better understanding of the tasks, activities, skills and operations that different kinds of game can offer and examine how these might match desired learning outcomes \citep{Connolly2012}.


\subsection{Game Elements}

(CHANGE IT UP) \citeauthor{Barnes2007} ran a project that made University Students create games that would teach basic programming. They carried out evaluations to test participant learning from the game, and made some interesting observations as follows: Clear instructions and game goals must be provided and accessible throughout the game, Learning goals must be clearly tied to in-game feedback that motivates the player (through, e.g. experience points, health), and penalizes guessing, Humor can be a motivation for in-game interaction \citep{Barnes2007}.

rewards remain controversial in education due to their possible negative side effects on individuals motivation, Despite the controversy concerning the use of external rewards in education (Cameron, Pierce, Banko, \& Gear, 2005). badges in the context of a technology-based innovation in an elementary school can enhance learning without undermining motivation \citep{Filsecker2014}.
 
 


\afterpage{\blankpage}

\chapter{Requirements Specification}
\afterpage{\blankpage}

\chapter{Design}
\afterpage{\blankpage}

\chapter{Implementation}
\afterpage{\blankpage}

\chapter{Testing}
\afterpage{\blankpage}

\chapter{Results}
\afterpage{\blankpage}

\chapter{Conclusions}
\afterpage{\blankpage}

\chapter{Future Work}

Here, it is important that not only the final outcomes are assessed, but also that the learning and training process itself is monitored continuously without impairing the playing/learning experiences (e.g. via psycho physiological measurements or automated logs/recordings of player behaviour). This is especially beneficial as it can inform new ways to make learning games more adaptive so that they can always offer help or additional information when the players need it (e.g. when they get stuck at a certain point of a game) \citep{Breuer2010}.

\afterpage{\blankpage}


\bibliography{litreview}
\addcontentsline{toc}{chapter}{Bibliography}


\appendix
\addtocontents{toc}{\protect\setcounter{tocdepth}{-1}}
\addtocontents{toc}{\protect\setcounter{tocdepth}{0}}
\appendixpage

\renewcommand\chaptername{Appendix}
%\chapter{Technical Specification and GA of the Skyseeker}\label{app:techspec}

\newpage
\chapter{Uncertainty Analysis} \label{app:errors}

\chapter{Screenshots} \label{app:screenshots}

\chapter{Ethics Checklist} \label{app:ethicschecklist}

\end{document}  